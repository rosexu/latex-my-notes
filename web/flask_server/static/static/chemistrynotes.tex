\documentclass[12pt]{article}

\title{title}
\author{Rose Xu}
\date{\today}

\begin{document}
\maketitle

This is the body of latex hola hell yeah

v
vwv-v__

. Surface modication is a useful approach for improvement of the materials because it can add the
dcsrred property of the materials on their surface without changing their physical property. In this seminar, new
polymers for surface modification developed in our group will be presented. The polymers were designed and
synthesized by controlled polymerization toward oil-repelling surface and anti-biofouling surface.

Highly efcient surface breeding of alkyl-uoroalkyl copolymer for internal melt additive

Addition of polymer additives into a base polymer is an efcient method for modication of the base
material. Fluoroalkyl group-containing polymer additive may reduce the surface energy and wettability of
substrates. Random and block copolymer consisting of alkyl and uoroalkyl units were synthesized by
controlled radical polymerization as internal melt additives. It was found that the block copolymer bled out
more effectively from the base substrate and gave higher contact angles than random copolymer.

Star polymer coatings for antibiofouling surfaces

Biomedical synthetic materials, such as PET, silicone, etc., are prone to adhesion of proteins, cells, and
bacteria, causing functional failures. One of the promising approaches is covering surfaces of the devices with
high dense polymer brush consisting of a hydrophilic polymer but it requires cumbersome chemical reactions
on the surface. We employed a star polymer as a coating material which can easily add an anti-adhesion

activity on the surface by a simple drop casting. These star polymer coatings prevent adhesion of proteins,
cells, and bacteria.

New blood compatible polymer coating based on tetrauoroethylene/vinyl alcohol copolymer
Poly(ethylene-covinyl alcohol) (pEVOlI) is known as a blood compatible polymer and employed as a

hemodialvzer membrane for more than 30 years. We examined poly(tetrafluoro ethyleneco-vinyl alcohol)

(p4FVOll) as a new blood compatible polymer. It was found that p4FVOH exhibited much higher inhibition

against platelet adhesion than pEVOH. The inhibition was close to that of poly(methacryloyl
phosphorylcholine) (pMPC) which is one of the most biocompatible polymer. Thus, p4FVOH is a promising

material as a new blood compatible material. WWW

Professor Ando earned his PhD in 2000 from Kyoto University AC
and has held his position at NAl ST since 2007. Chemistry for Life

 

5:: a am

e nnnnn ma i, am ACS POLY/PMSE Student Chapter at the ACS POLY/PMSE student Chapter

\end{document}

