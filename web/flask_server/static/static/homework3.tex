\documentclass[12pt]{article}

\author{Rose Xu}
\date{\today}

\begin{document}
\maketitle

HW 1:
1. Revised Exercise 1.28: Human measurements provide a rich area of application. . . (page 27)
a. Construct a histogram and boxplot for the IQ scores of the 33 students. Which graphical
display do you prefer? Explain.
b. Calculate summary statistics for the IQ scores. Include the sample mean, sample standard
deviation, median, interquartile range (IQR), and ve-number summary.
c. Should you use the mean and standard deviation or the median and interquartile range (IQR) to
describe the center and variability, respectively, of the IQ scores? Explain.

2. Revised Exercise 1.34: Exposure to microbial products, especially endotoxin. . . (page 34)

a. Construct two histograms for the concentration (EU/mg) in settled dust, one for the sample of
urban homes and one for the sample of farm homes.

b. Calculate summary statistics for the concentration (EU/mg) in settled dust for each of the
samples. Include the sample mean, sample standard deviation, median, interquartile range
(IQR), and ve-number summary.

c. Compare the two samples (urban homes and farm homes) with respect to their histograms and
summary statistics. Be sure to address shape, center, variability, and any unusual features.

3. Exercise 1.70: Elevated energy consumption during exercise continues after... (page 47)

4. Revised Exercise 1.72: Anxiety disorders and symptoms can often be effectively... (page 47)
a. Construct a histogram for the receptor binding measure for individuals suffering from PTSD
and for healthy individuals.
b. Calculate summary statistics for the receptor binding measure for individuals suffering from
PTSD and for healthy individuals.
c. Using your results from parts (a) and (b), compare the distribution of receptor binding measure
for individuals suffering from PTSD and to that for healthy individuals.

\end{document}

